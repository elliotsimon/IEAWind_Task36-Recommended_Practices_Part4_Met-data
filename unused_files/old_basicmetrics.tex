\chapter{Basic Statistical Metrics and their Meaning in Wind Power Forecasting}\label{ch:metrics}
%(Jakob, John)

Forecast evaluation is widely used in the power industry with important applications such as quality checks of operational forecasts, forecast trials and benchmarks and calculating performance incentives.
Despite its importance, evaluation has not received much attention in literature, and those publications that deal with evaluation methods and metrics are often written in the context of model development and thus rather technical and not very practically oriented for industry applications.
This guideline focuses therefore on the Forecast Users perspective rather than on that of model developers in order to raise awareness on the importance of appropriate evaluation and to avoid common pitfalls when evaluating wind power forecasts. The goal of this section is therefore to provide a reference and strategic paths for the industry to set up meaningful evaluation frameworks.

\section{Literature Review of evaluation metrics for the power industry}\label{sec:LiteratureReview}

Examples are Madsen et al. 2005 proposed in 2005 standard protocols for deterministic forecast evaluation, Bessa et al. 2010 discussed the relationship between forecast quality and value. 
Pinson and Girard 2012 discussed evaluation approaches for wind power scenario forecasts.

\cite{lerch}
\cite{murphy}
\cite{wilks}
\cite{zhang}

\cite{madsen2005} recommended a 

In \ref{madsen2005}, it is suggested to start by defining an ``operational  framework'' that includes a specification of:
\begin{itemize}
    \item Installed capacity. Number and type of wind turbines.
    \item Horizon of predictions (1, 2, ..., 48, .. hours ahead).
    \item Use  of  on-line  measurements  as  input.  Specify  which  data  are  used  (e.g.  power production, wind speed, etc.).
    \item Sampling  strategy. \\ 
        Specify  whether  the  measurements  are  instant  readings  or  the
    average over some time period, e.g. the last 10 minutes before the time stamp. This
    should be specified for all observed variables.
    \item Characteristics of NWP forecasts \\
        frequency of delivery, delay in delivery, horizon, time step, resolution, grid values or interpolated at the position of the farm.
    \item  Frequency  of  updates  of  the  prediction  model \\
        Actually,  some  models  only  give forecasts  when  new  NWPs  are  delivered  (i.e.  every  6,  12  or  24  hours)  while  some others  operate  with  a  sliding  window  (typically  one  hour)  since  they  consider  on-line data as input.
\end{itemize}



….TO BE COMPLETED...
