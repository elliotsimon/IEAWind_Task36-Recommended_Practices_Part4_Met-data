


1. Resource Assessment vs Forecasting Measurements: add section that discusses the differences between deploying and operating measurement systems for these two types of applications.
         Maybe an additional section under 1.3 (Use and Application of Measurements)?

 2. Representativeness of measurements:  There isn’t a section that explicitly addresses this issue.
         Certainly 3.1 and 3.2 are very relevant to this issue. Perhaps we could restructure this part to be:
        3.1 Representativeness of Measurements
        3.1.1 Selection of location
        3.1.2 Selection of height
        3.1.3 Measurement Characteristics of Different Technologies

        --> Add an introductory paragraph under 3.1 to describe the issue - that the objective is to make measurements that are well correlated to the power generation (wind or solar) and what issues affect that correlation. 

        --> discuss horizontal and vertical location specifics in 3.1.1 and 3.1.2.

        --> in 3.1.3 discussion of conceptual differences in measurements from anemometers on met masts (point measurements), remote sensing devices
        (lidar,sodar - which are some type of volume average) and nacelle mounted anemometers (very representative of height and location of turbines but obstructed). What do you thinks?

 3. Examples of System Operator Met Measurement Requirements:  ---> collect a few examples and reference or establish a table that compares some of the requirements of some of the major
           system operators in North America and Europe (and maybe elsewhere if we can find the info)?
     --> separate section - Section 7 or as an appendix?