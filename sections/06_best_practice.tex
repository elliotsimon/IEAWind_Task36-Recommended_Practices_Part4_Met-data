\chapter{Best Practice Recommendations {\color{magenta}{Contributing author: }}}
%\\ \color{green} POST PUBLIC REVIEW VERSION\color{black}}
\label{ch:bestpractice}

\noindent\begin{tcolorbox}
\parbox{\textwidth}{
\emph{\textbf{Key Points}\\
The recommendations in this section are based on the following set of principles:
\begin{itemize}
    \item 
    \item 
    \item 
    \item 
\end{itemize}
}}
\end{tcolorbox}

In this last chapter, the principles developed in the previous chapters are brought to the application level. In other words, the somewhat theoretical considerations from the previous chapters are now applied to real-world problems. 

%%%%%%%%%%%%%%%%%%%%%%%%%%%%%%%%%%%%%%%%%%%%%%%%%%%%%%%%%%%%%%%%%%%%%%%%%%%%%%%%%%%%%%%%%%%%%

\noindent\begin{tcolorbox}
\parbox{\textwidth}{
\emph{\textbf{Key Points}\\
Selection of instrumentation ...
}}
\end{tcolorbox}


\section{Instrumentation {\color{magenta}{Contributing author: }}}
 
    \subsection{Definitions }
 
    \subsection{Choice of instruments }
 
    \subsection{Verification of instrument signals }
    
     \subsubsection{Plausibility Analyses}\label{sec:plausibilityanalaysis}

\section{Requirements for the implementation into power grid operation {\color{magenta}{Contributing author: COM}}}
    
    \subsection{Implementation and Testing Rules }
    
    \subsection{Validation and Verification }
    
    \subsection{Maintenance Schedules }
    
\section{Requirements for the implementation into power plant operation {\color{magenta}{Contributing author: }}}
    \subsection{Implementation and Testing Rules }
    
    \subsection{Validation and Verification }
    
    \subsection{Maintenance Schedules }

\section{Quality Requirements {\color{magenta}{Contributing author: COM}} }\label{sec:data_quality_requirements}
    
    
    \subsection{Valid ranges }\label{subsec:validranges}
    
    
    \subsection{Valid error levels }\label{subsec:validerrors}


