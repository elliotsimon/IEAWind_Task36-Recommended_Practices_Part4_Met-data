\chapter{Background and Objectives {\color{magenta}{Contributing author: COM, JY}}} \label{ch:introduction}
%%\color{green} POST PUBLIC REVIEW VERSION \\ \color{black}

\noindent
\begin{comment}
\begin{tcolorbox}
\parbox{\textwidth}{
\emph{\textbf{Key Points}\\
This is the fourth part of a series of four recommended practice documents that deals with the development and operation of renewable energy forecasting solutions for the power market.
This part provides information and guidelines for the best practices of met measurements in the real-time operation of grid operators....
}}
\end{tcolorbox}
\end{comment}


\section{BEFORE YOU START READING}

This is the forth part of a series of four recommended practice documents that deal with the development and operation of forecasting solutions in the power market. 
The first part  “Forecast Solution Selection Process” deals with the selection and background information necessary to collect and evaluate when developing or renewing a forecasting solution for the power market. 
The second part “Design and Execution of Benchmarks and Trials”, of the series deal with benchmarks and trials in order to test or evaluate different forecasting solutions against each other and the fit-for-purpose. 
The third part “Forecast Solution Evaluation”, which is the current document, provides information and guidelines regarding effective evaluation of forecasts, forecast solutions and benchmarks and trials.


\section{Introduction {\color{magenta}{Contributing author: COM}}}

Meteorological measurements provide an independent measure of the wind resource and weather situation at any given time. This information can and is, as technology enhances, not only an obligation that stems from technical requirements of the system operator, but is also used to optimise the operation of wind turbines by the wind farm operators.
For both the system operator and the wind farm operator, these measurements are an independent signal at the wind farm that can warn about critical weather and provide an indication on whether the wind turbines work at their expected performance level. For the transmission system operator, such measurements can additionally be used for situational awareness of the weather in the control area that may affect the transmission lines. They also provide a second means of verification, whether the power signal at a given wind farm is malfunctioning in situations that may be critical in terms of grid operation.

Data assimilation with independent measurements from wind farms have also been tested by meteorological centres (e.g. \cite{Marquis2012,EWELINE2011}). One of the most important findings so far is that the quality of data provided is the most essential issue to be solved in order to gain higher quality forecasts with such measurements. In other words, if there is no specific effort put into standardisation of requirements in the power industry, the benefits can not be achieved....

This recommended practice (RP) is aimed to provide background information on meteorological instrumentation, their recommended setup, maintenance, quality control and use in the real-time environment of wind and solar power generation plants. The document also provides practical guidelines for quality control and recommended standards to be applied in the setup and operational phase of wind and solar projects for plant operators, system operators and power traders or balance responsible parties .... 


\section{Use and Application of {\color{blue}{(real-time ?)}} Measurements{\color{magenta}{Contributing author: JY, JZ}}}

\subsection{Resource Assessment versus Forecasting Measurements}
%This section discusses the differences between deploying and operating measurement systems for these two types of applications.
\subsection{System Operation}

\subsection{Power plant operation}

\subsubsection{Wind plant operation}

\subsubsection{Solar/PV plant operation}

\subsection{Power trading in electricity markets {\color{magenta}{Contributing author: ES}}}

\section{Available {\color{blue}{(Applicable ?)}} Standards {\color{magenta}{Contributing author: COM}}}

\subsection{Standards and Guidelines for Wind Measurements}

\subsection{Standards and Guidelines for Solar/PV Measurements}
