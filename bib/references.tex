%\begin{thebibliography}{27}
\chapter*{References}\label{references}
\setlength{\parindent}{-1cm}\\

%\bibitem{AESO2011a}
\noindent
%AESO, ISO rule 502-1: Wind Aggregated Generating Facilities Technical Requirements, 2011. Online:\small{\url{http://www.aeso.ca/downloads/502_1_Wind_Aggregated_Generating_Facilities_Technical_Requirements.pdf}} , \small{\url{http://www.aeso.ca/rulesprocedures/18592.html}} => Division 502-Technical Requirements.  

%\bibitem{AESO2011b}
AESO ISO rule 502-1: Wind Aggregated Generating Facilities Technical Requirements - Wind Forecasting Information. Online: \small{\url{http://www.aeso.ca/rulesprocedures/18592.html}} => Division 502 - 11-007R Wind Power Forecasting, \small{\url{http://www.aeso.ca/downloads/2011-06-23_Wind_Forecasting_ID.pdf}}

Allik, A. Uiga, J., Annuk, A.,  Deviations between wind speed data measured with nacelle-mounted anemometers on small wind turbines and anemometers mounted on measuring masts, Agronomy Research 12(2), 433­444, 2014.                      
Online: \small{\url{http://www.tuuleenergia.ee/wp-content/uploads/BSE-2014-renewable-energy\\-artikkel.pdf}}


Angelou, Nikolas et al. Doppler lidar mounted on a wind turbine nacelle – UPWIND deliverable D6.7.1 Roskilde: Danmarks Tekniske Universitet, Risø Nationallaboratoriet for Bæredygtig Energi. 2010. (Denmark. Forskningscentre Risoe. Risoe-R; Journal number 1757(EN)). 
Online: \small{\url{http://orbit.dtu.dk/en/publications/}}
%doppler-lidar-mounted-on-a-wind-turbine-nacelle--upwind-deliverable-d671(cde3bbcc-0663-4421-b75f-c976fdbc49a4).html"}


Ammonit Wind measurement. Online 2016-09-06:\\                          
\small{\url{http://www.ammonit.com/en/wind/wind-measurement}}


Ammonit, AQ 510 Classification Test, Online taken on 12.09.2016. Online: \small{\url{http://www.ammonit.com/en/news/news/521-aq510-classification-test}}


Antoniou I, Jørgensen HE, Mikkelsen T, Frandsen S, Barthelmie R, Perstrup C, Hurtig M. Offshore wind profile measurements from remote sensing instruments. In Proceedings of the European Wind Energy Association Conference and Exhibition. Athens, 2006.

Basu, S., Porté-Agel, F., Foufoula-Georgiou, E., Vinuesa, J-F. and Pahlow, M. (2006) Revisiting the Local Scaling Hypothesis in Stably Stratified Atmospheric Boundary-Layer Turbulence: an Integration of Field and Laboratory Measurements with Large-Eddy Simulations. Boundary-Layer Meteorology 18(3): 473-500.          
Online: \small{\url{http://dx.doi.org/10.1007/s10546-005-9036-2}}

Berg, L.K., M. Pekour, and D. Nelson: Description of the Columbia Basin Wind Energy Study (CBWES). Pacific Northwest National Laboratory Tech. Rep. PNNL-22036, 14 pp., 2012. Online: \small{\url{http://www.pnnl.gov/main/publications/external/technical_reports/PNNL-22036.pdf}}

Bing{\"o}l, F., Mann, J., Foussekis, D.: Conically scanning lidar error in complex terrain. Meteorologische Zeitschrift, 18 (2), pp. 189-195(7). Emeis, Stefan; Harris, Michael; Banta, Robert M., 2007: Boundary-layer anemometry by optical remote sensing for wind energy applications. Meteorologische Zeitschrift, 16 (4), pp. 337-347(11),2009a.

Bing{\"o}l, F., Complex Terrain and Wind Lidars. Ph.d. thesis, Risø DTU, 2009. Online: \small{\url{http://orbit.dtu.dk/fedora/objects/orbit:83301/datastreams/file_5245709/content}}

Blackadar, A. K., Boundary layer wind maxima and their significance for the growth of nocturnal inversions. Bull. Amer. Meteor. Soc., 38, 283-290, 1957. Online: \small{\url{http://twister.ou.edu/MM2005/Blackardar1957BAMS.pdf}}

BPA, Bonneville Power Adminstration's Technical Requirements for Interconnection to the BPA Transmission Grid, STD-N-000001, 2015. Online: \small{\url{https://www.bpa.gov/transmission/Doing\%20Business/Interconnection/Pages/default.aspx BPA_tech_requirements_interconnection.pdf}}

Bradley, S., Wind speed errors for LIDARs and SODARs in complex terrain; 14th International Symposium for the Ad

Bradley, S. Wind speed errors for lidars and sodars in complex terrain. IOP Conference Series: Earth and Environmental Science, 1(1):012061, 2008b. Online: \small{\url{http://stacks.iop.org/1755-1315/1/i=1/a=012061}}

Bradley, S., Y. Perrott, P. Behrens, and A. Oldroyd. Corrections for wind-speed errors from sodar and lidar in complex terrain. Boundary-Layer Meteorology, 143:37­48, 2012a. 
Online: \small{\url{http://dx.doi.org/10.1007/s10546-012-9702-0}}

Bradley, S., S. von Hünerbein, and T. Mikkelsen. A bistatic sodar for precision wind profiling in complex terrain. Journal of Atmospheric and Oceanic Technology, 29(8):1052­1061, 2012b. Online: \small{\url{http://dx.doi.org/10.1175/JTECH-D-11-00035.1}}

Burin des Roziers, E., REPEATABILITY OF ZEPHIR PERFORMANCE DEMONSTRATED ACROSS A SAMPLE OF MORE THAN 170 IEC COMPLIANT VERIFICATIONS, Report by ZephirLiDAR, 2014.  Online: \small{\url{http://www.ammonit.com/images/stories/download-pdfs/TestReports/Repeatability_of\\_ZephIR_performance.pdf}}

California ISO, Wind Technical Requirements, fifth Replacement California ISO Tariff archive, Appendix Q Eligible Intermittent Resources Protoco EIRP, 2014.   Online: \small{\url{http://www.caiso.com/rules//Pages/Regulatory/TariffArchive/Default.aspx}},\\ CALISO\_PIRP\_WindTechnicalRequirements02-Mar-2009.pdf,\\ AppendixQ\_EligibleIntermittentResourcesProtocolEIRP\_May1\_2014.pdf.


California ISO,  Business Practice Manual for Direct Telemetry, Version 9, p.57, May 2016. Online: \small{\url{http://www.caiso.com/rules/Pages/BusinessPracticeManuals/Default.aspx}}\\ -> Dokument: BPM\_for\_Direct\_Telemetry\_V9\_Clean.docx

Campbell, I., A Comparison of Remote Sensing Device Performance at Rotsea, Report by Renewable Energy Systems, Document Reference: 01485-000090 Issue: 05, 2011.
Online: \url{https://www.renewablenrgsystems.com/services-support/resources/documentation-and-downloads/white-papers/detail/a-comparison-of-remote-sensing-device-performance-at-rotsea}
	
Courtney, M., Wagner, R., Lindelöw, P., Commercial Lidar profilers for wind energy. A comparative guide. Proc. of European Wind Energy Conf., Brussels, 2008.
Online: \url{https://www.renewablenrgsystems.com/services-support/resources/documentation-and-downloads/white-papers/detail/commercial-\\lidar-profilers-for-wind-energy-a-comparative-guide},\\
\small\url{http://www.renewablenrgsystems.com/assets/resources/Commercial-Lidar-Profilers-for-Wind-Energy-Whitepaper.pdf}


Cuerva A., Sanz-Andrés A., Franchini S., Eecen P., Busche P., Pedersen T.F., Mouzakis F., "ACCUWIND ­ Accurate Wind Measurements in Wind Energy, Task 2. Improve the Accuracy of Sonic Anemometers, Final Report", UPM/IDR Madrid June 2006


Dahlberg J.-Å., Pedersen T.F., Busche P., "ACCUWIND ­ Methods for Classification of Cup Anemometers", Risø-R-1555(EN), May 2006.


Dahlberg, J-A., Frandsen, S. T., Aagaard Madsen, H., Antoniou, I., Friis Pedersen, T., Hunter, R., Klug, H., Is the nacelle mounted anemometer an acceptable option in performance testing? In E. L. Petersen, P. Hjuler 

Deutsche Windgard, Performance Verification of Galion, Report 13001, 2013. Online: \url{http://www.sgurrenergy.com/wp-content/uploads/2015/09/DWGGalion-Report.pdf}


DNVGL, Review of the Spinner anemometer from ROMO Wind iSpin, Report No.: 113605-DKAR-R-01, Rev. 3, Document No.: 113605-DKAR-R-01, 2015.           
Online: \small{\url{http://romowind.com/media/113605-DKAR-R-01\\ -Rev-3-2015-03-06.pdf}}


Drechsel, S., G. J. Mayr, J. W. Messner, R. Stauffer, Wind Speeds at Heights Crucial for Wind Energy: Measurements and Verification of Forecasts, J. Appl. Meteor. Climatol., 51, 1602-1617, 2012.


Eecen P.J., Mouzakis F., Cuerva, A "ACCUWIND, Work Package 3 Final Report", ECN-C-06-047, May 2006.


Emeis, S., Harris,M., Banta, R.M.,  Boundary-layer anemometry by optical remote sensing for Wind energy applications. Meteorologische Zeitschrift, 16(4):337{347, 2007.


Enevoldsen, K., Comparison of 3D turbulence measurements using three staring wind lidars and a sonic anemometer. IOP Conference Series: Earth and Environmental Science, 1(1):012012, 2008.
Online \small{\url{http://stacks.iop.org/1755-1315/1/i=1/a=012012}}


Environmental Protection Agency (EPA), Office of Air Quality, Planning and Standards, Meteorological Monitoring Guidance for Regulatory modelling Applications, Doc. EPA-454/R-99-005, 2000. Online:\small{\url{https://www3.epa.gov/scram001/guidance/met/mmgrma.pdf}}


ERCOT, ERCOT Business Practices: ERCOT AND QSE OPERATIONS PRACTICES DURING THE OPERATING HOUR, Version 5.5, April 2012.                              
Online:\url{http://www.ercot.com/mktrules/bpm} -> \\BP\_ERCOT\_And\_QSE\_Operations_Practices_v5_5.doc


EWEA, Wind Energy: The Facts Homepage, Best Practice for Accurate Wind Speed Measurements, Online extracted August 2016.                        
Online:\small{\url{http://www.wind-energy-the-facts.org/best-practice-for-accurate-\\wind-speed-measurements.html}}

\bibitem{EWELINE2011}
EWeLINE project: Development of innovative weather and power forecast models for the grid integration of weather dependent energy sources.      
Online: \small{\url{http://www.projekt-eweline.de/en/project.html}}


Falbe-Hansen, GL Garrad Hassan, Technology Review of the ROMO Wind Spinner Anemometer I, Document No. 111789-DKHI-R-01, 2012.
Online: \small{\url{http://romowind.com/media/GL-Garrad-Hassan-spinner-\\anemometer-report.pdf}}


Froese, M., Vattenfall and ROMO Wind publish performance data verifying accuracy of iSpin technology, Windenergy Enegineering&Development, Article from 25.May 2016. Online:\small{\url{http://www.windpowerengineering.com/featured/business-news-projects/vattenfall-romo\\-wind-publish-performance-data-\\verifying-accuracy-ispin-technology/2016_Froese_Windenergy-Eng-and-Dev_Romowind-\\Vattenfall-experiment-iSpin.pdf}}


G{\"o}{\cc}men T., Giebel G., Data-driven Wake Modelling for Reduced Uncertainties in short-term Possible Power Estimation: Paper. Journal of Physics: Conference Series, 1037(7), 072002, DOI:10.1088/1742-6596/1037/7/072002, 2018.


G{\"o}{\cc}men, T., Giebel, G., Estimation of turbulence intensity using rotor effective wind speed in Lillgrund and Horns Rev-I offshore wind farms, Renewable Energy, Vol. 99. pp. 524-532, 2016.


Grund, C.J.; Banta, R.M.; George, J.L.; Howell, J.N.; Post, M.J.; Richter, R.A.; Weickman, A.M., High-Resolution Doppler Lidar for Boundary Layer and Cloud Research, J.Atmospheric & Oceanic Technology (18), pp.376-393, 2001.


Hasager, Charlotte B.; Stein, Detlef; Courtney, Michael; Peña, Alfredo; Mikkelsen, Torben; Stickland, Matthew; Oldroyd, Andrew. 2013. "Hub Height Ocean Winds over the North Sea Observed by the NORSEWInD Lidar Array: Measuring Techniques, Quality Control and Data Management." Remote Sens. 5, no. 9: 4280-4303.


HECO, Hawaiian Electric Company, personal communication, October 2016.

IEA Wind Task 36: Wind Power Forecasting, 2016. Online: \url{http://www.ieawindforecasting.dk/}


IEA Wind Task 11: Best Technology Information Exchange Recommended Practices, Online: \small{\url{http://www.ieawind.org/task_11/recomend_pract.html}}


IEA Wind Task 11, Guideline 11  WIND SPEED MEASUREMENT AND USE OF CUP ANEMOMETRY, Second Edition, 2003.
Online: \small{\url{http://www.ieawind.org/task_11/recommended_pract/Recommended\\\%20Practice\%2011\%20Anemometry_secondPrint.pdf}}


IEA Wind Task 11, Guideline 15:  GROUND-BASED VERTICALLY-PROFILING REMOTE SENSING FOR WIND RESOURCE ASSESSMENT, First Edition, Jan. 2013.
Online: \small{\url{http://www.ieawind.org/task_11/recommended_pract/Recommended\%20Practice\%2015\%20RemoteSe\\nsing\%201stEd.pdf}}


IEC Standard 61400-12-1:2005(E)   “Wind   turbines   -   Part   12-1:   Power   perfor-mance measurements of electricity producing wind turbines, First edition 2005-12/ Annex F, “Cup anemometer calibration procedure”, 2005


IEC Standard 61400-12-1:2017, “Power performance measurements of electricity producing wind turbine”, 2017.


IEC/ISO 17025:2005(E), General requirements for the competence of testing and calibration laboratories, Version 2, 2005.
Joint Committee for Guides in Metrology, Evaluation of measurement data -- An introduction to the "Guide to the expression of uncertainty in measurement" and related documents, JCGM 104:2009, 2009.   
Online: \small{\url{https://en.wikipedia.org/wiki/Joint_Committee_for_Guides_in_Metrology}}

Jensen, K. Rave, P. Helm, & H. Ehmann (Eds.), Wind energy for the next millennium. Proceedings. (pp. 624-627). London: James and James Science Publishers, 1999.          
Online:\small{\url{https://www.etde.org/etdeweb/servlets/purl/679606/?type=download}}  (see pp.55ff).

Joint Committee for Guides in Metrology, Evaluation of measurement data -- Guide to the expression of uncertainty in measurement. JCGM 100:2008, 2008.


Joint Committee for Guides in Metrology, Evaluation of measurement data -- Supplement 1 to the "Guide to the expression of uncertainty in measurement" -- Propagation of distributions using a Monte Carlo method, JCGM 101:2008, 2008a.


Joint Committee for Guides in Metrology, Evaluation of measurement data -- Supplement 2 to the "Guide to the expression of uncertainty in measurement" -- Propagation of distributions using a Monte Carlo method, JCGM 102:2011, 2011.


Joint Committee for Guides in Metrology, Evaluation of measurement data -- The role of measurement uncertainty in conformity assessment, JCGM 106:2012, 2012.


Jones, L.E., Strategies and Decision Support Systems for Integrating Variable Energy Resources in Control centres for Reliable Grid Operations, Global Best Practices, Examples of Excellence and Lessons Learned, Report for the Department of energy under the Award Number DE-EE0001375, 2012.


Kelley, N.D., Jonkman, B.J., Scott G.N., Pichugina, Y.L., Comparing Pulsed Doppler LIDAR with SODAR and Direct Measurements for Wind Assessment, Proc. AWEA Wind Conference 2007, Ref. NREL/CP-500-41792, July, 2007. 
Online: \small{\url{http://www.nrel.gov/wind/pdf}s/41792.pdf}}


Kelly, N. D., M. Shirazi, D. Jager, S. Wilde, J. Adams, M. Buhl, P. Sullivan, and E. Patton, 2004: Lamar low-level jet program interim report. NREL/TP-500-34593. 
Online at \small{\url{http://www.nrel.gov/docs/fy04osti/34593.pdf}}


Kindler, D., Oldroyd, A., Macaskill, A., Finch, D., An eight month test campaign of the Qinetiq ZephIR system:  Preliminary results. Meteorol. Z.,  16(5):479-489, 2007.


Krishnamurthy, R., Choukulkar, A., Calhoun, R., Fine, J., Oliver, A. and Barr, K.S., Coherent Doppler lidar for wind farm characterization. Wind Energ., 16: 189-206. DOI: 10.1002/we.539, 2013.


Lang, S., McKeogh, E., LIDAR and SODAR Measurements of Wind Speed and Direction in Upland Terrain for Wind Energy Purposes, Remote Sens. 2011, 3, 1871-1901; doi:10.3390/rs3091871, 2011. Online: \url{http://www.mdpi.com/journal/remotesensing}


LitGRID, PERDAVIMOS SISTEMOS OPERATORIAUS TIPINĖS IŠANKSTINĖS PROJEKTAVIMO SĄLYGOS VĖJO ELEKTRINIŲ PARKO PRIJUNGIMUI PRIE 110 kV ĮTAMPOS ELEKTROS PERDAVIMO TINKLO, Nr. 47, 2010.
Online:\url{http://www.litgrid.eu/index.php/services/connection-to-the-\\transmission-grid-/585}


Lundquist, J., K. Lundquist, Eugene S. Takle, Matthieu Boquet, Branko Kosovi, Michael E., Rhodes1, Daniel Rajewski3, Russell Doorenbos, Samantha Irvin, Matthew L. Aitken1, Katja, Friedrich1, Paul T. Quelet1, Jiwan 
Rana1, Clara St. Martin1, Brian Vanderwende, 1Rochelle, Worsnop1, Lidar observations of interacting wind turbine wakes in an onshore wind farm, Renewable NRG Systems Whitepaper, Online September 2016. 
Online:\\\small{\url{https://www.renewablenrgsystems.com/services-support/resources/documentation-and-\\downloads/white-papers/detail/lidar-observations-of-interacting-wind-turbine-wakes\\-in-an-onshore-wind-farm}}

\bibitem{Marquis2012}
Marquis, M., Wilczak, J., Finley, C., Freedman, J., Wind Forecasting Improvement Project (WFIP) final report, National centre of Oceanographic and Atmospheric Administration (NOAA), 2014. 
Online: \url{http://energy.gov/sites/prod/files/2014/05/f15/wfipandnoaafinalreport.pdf} Project Homepage:  
\url{http://www.esrl.noaa.gov/psd/renewable_energy/wfip/}


Masson C, Smaili A, "Numerical Study of Turbulent Flow around a Wind Turbine Nacelle", Wind Energy, 9:281-298, 2006.


Matzler, C., Thermal Microwave Radiation: Applications for Remote Sensing, The Institution of Engineering and Technology, London, Chapter 1, 2006

MEASNET, Cup Anemometer Calibration Procedure – Version 2, October 2009.
Online: \url{http://www.measnet.com/wp-content/uploads/2011/06/measnet_anemometer_calibration_v2_oct_2009.pdf}


MEASNET, Evaluation of Site specific Wind conditions, Version 2, April 2016. 
Online:\url{http://www.measnet.com/wp-content/uploads/2016/05/Measnet_SiteAssessment_V2.0.pdf}


M{"\o}hrlen, C., J{\o}rgensen, J.U., A new algorithm for Upscaling and Short-term forecasting of wind power using Ensemble forecasts, Proc. of the 8th International Workshop on Large-Scale Integration of Wind Power into Power Systems as well as on Transmission Networks for Offshore Wind Farms, pp., 2009.                          
Online: \url{http://www.weprog.com/files/weprog_windintegration_2009_p54_paper.pdf}


M{\"o}hrlen, C., Uncertainty in Wind Energy Forecasting, PhD Thesis, pp19-21, 2004.
Online: \url{http://library.ucc.ie/record=b1501384~S0} full text: \url{http://hdl.handle.net/10468/193} 

Monserrat, S. and Thorpe, A. J. , Gravity-Wave Observations Using an Array of Microbarographs In the Alearic Islands. Q.J.R. Meteorol. Soc., 118: 259–282. doi:10.1002/qj.49711850405, 1992.


Morris, VR, Ceilometer Instrument Handbook, DOE/SC-ARM-TR-020, 2016.      Online: \url{http://www.arm.gov/publications/tech_reports/handbooks/ceil_handbook.pdf}

Morton, B., Osler, E., Coubart-Millet, G., Wind Iris Operational Applications, RNRG Whitepaper, 2016.
Online: \small{\url{https://www.renewablenrgsystems.com/services-support/resources/documentation-and-\\downloads/white-papers/detail/white-paper-wind-iris-operat\\ional-applications}}

NYISO, Wind Plant Operator Forecast Data Guide, June 2016. 	Online: \small{\url{http://www.nyiso.com/public/webdocs/markets_operations/documents/Manuals_and_Guides/Guides/User_Guides/Wind_Plant\\_Operator_Forecast_Data_Guide.pdf}}


Nakafuji, D., Making the elephant fly: Integrating distributed energy resources onto the grid in Hawaii, Guest Comment, published Online at Smart Electric Power Alliance webpage on 15.September 2016.
Online: \small{\url{http://www.solarelectricpower.org/utility-solar-blog/2016/september/making-the-elephant-\\fly-integrating-distributed-energy-resour\\ces-onto-the-grid-in-hawaii.aspx}}


Pedersen, T.F., Sørensen, N.,Enevoldsen, P., Aerodynamics and Characteristics of a Spinner Anemometer, Risø National Laboratory DTU Institute of Physics and IOP Publishing Limited, Doi: dx.doi.org/10.1088/1742-6596/75/1/012018, 2007. 
Online: \small{\url{http://www.risoe.dk/rispubl/art/2007_247.pdf}}


Pedersen, T.A., HA Madsen, R Møller, M Courtney, N Sørensen, Spinner anemometry - basic principles for application of the technology, 2016. 

Online: \small{\url{http://romowind.com/media/Spinner-anemometry-a-scientific-paper.pdf}}


Popinet, S., Murray Smith, and Craig Stevens, 2004: Experimental and Numerical Study of the Turbulence Characteristics of Airflow around a Research Vessel. J. Atmos. Oceanic Technol., 21, 1575–1589, doi: 10.1175/1520-0426(2004)021<1575:EANSOT> 2.0.CO;2.                                                                 

Pedersen T.F., Dahlberg J.-Å., Busche P., "ACCUWIND ­ Classification of Five Cup Anemometers According to IEC61400-12-1", Risø-R-1556(EN), May 2006.


Pedersen, F., Demurtas,T.G., Zahle, F., Calibration of a spinner anemometer for yaw misalignment measurements. Wind Energy, 18, 1933–1952. DOI: 10.1002/we.1798, 2015.
Online: \small{\url{http://orbit.dtu.dk/ws/files/99931702/Calibration_of_a_spinner_anemometer.pdf}}


Pe{\~n}a, A., Hasager, C. B., Lange, J., Anger, J., Badger, M., Bing{\"o}l, F.,.. Würth, I., Remote Sensing for Wind Energy. DTU Wind Energy.  (DTU Wind Energy E; No. 0029(EN)), 2013. Online: \small{\url{http://orbit.dtu.dk/files/55501125/Remote_Sensing_for_Wind_Energy.pdf}}


Pichugina, Yelena L., Robert M. Banta, W. Alan Brewer, Scott P. Sandberg, R. Michael Hardesty, Doppler Lidar-Based Wind-Profile Measurement System for Offshore Wind-Energy and Other Marine Boundary Layer Applications. J. Appl. Meteor. Climatol., 51, 327-349.DOI: 10.1175/JAMC-D-11-040.1, 2012.


Pichugina, Yelena L., Sara C. Tucker, Robert M. Banta, W. Alan Brewer, Neil D. Kelley, Bonnie J. Jonkman, Rob K. Newsom, 2008: Horizontal-Velocity and Variance Measurements in the Stable Boundary Layer Using Doppler Lidar: Sensitivity to Averaging Procedures. J. Atmos. Oceanic Technol., 25, 1307-1327. DOI: 10.1175/2008JTECHA988.1, 2008.


Pinson, P., and R. Hagedorn, Verification of the ECMWF ensemble forecasts of wind speed against observations. Meteor. Appl.,1-20, doi:10.1002/met.283, 2012. Online:\\\small{\url{http://www.ecmwf.int/en/elibrary/11679-verification-ecmwf-ensemble-forecasts-\\wind-speed-against-observations}}


PJM, Generator Operational Requirements, Manual 14D, Rev. 38, Effective Date: June 1, 2016. 	Online:\small{\url{http://www.pjm.com/documents/manuals.aspx}}


ROMOWIND, Knowledge Base: Free Data Access to the Noerrekaer Windfarm, Online (2016-09-05): Online: \url{http://www.romowind.com/en/open-dataScintec}, Datasheet Sodar wind profiler MFAS, Version 2012. 
Online: \url{http://www.scintec.com/english/Web/scintec/Details/A032002.aspx}


Scipi{\'o}n, D., Palmer, R., Chilson, P., Fedorovich, E., Botnick, A., Retrieval of convective boundary layer wind field statistics from radar profiler measurements in conjunction with large eddy simulation, Meteorologische Zeitschrift, Volume 18, Number 2,  pp. 175-187(13), 2009. 


Sj{\"o}holm,M., Mikkelsen, T., Mann, J., Enevoldsen, I and Courtney, M.,Time series analysis of continuous-wave coherent Doppler Lidar wind measurements. IOP Conf. Ser.: Earth Environ. Sci., 1, 012051. DOI: 10.1088/1755-1315/1/1/012051, 2008.


Smith DA, Harris M, Coffey AS, Mikkelsen T, Jørgensen HE, Mann J, Danielian R. Wind lidar evaluation at the Danish wind test site in Høvsøre. Wind Energy 2006; 9:87-93.


Smith, B., Link, H., Randall, G., McCoy, T., Applicability of Nacelle Anemometer Measurements for Use in Turbine Power Performance Tests, Proc. American Wind Energy Association Windpower Conference, NREL Report NREL/CP-500-32494, 2002. Online: \url{http://www.nrel.gov/docs/fy02osti/32494.pdf}


Tordoff, S., How to plan the perfect wind measurement campaign, Wind Power Monthly Online, March, 1, 2013. Online: \url{http://www.windpowermonthly.com/article/1172038/plan-\\perfect-wind-measurement-campaign}


Ulaby, F.T.,Moore,R.K. Fung. A.K., Microwave Remote Sensing—Active and Passive”. By F. T. Ulaby. R. K. Moore and A. K. Fung. (Reading, Massachusetts: Addison-Wesley,  Volume I: Microwave Remote Sensing Fundamentals and Radiometry, 1981.


Wagner, R., and Davoust, S., Nacelle lidar for power curve measurement  - Avedøre campaign. DTU Wind Energy.  (DTU Wind Energy E; No. 0016), 2013. Online: \url{http://orbit.dtu.dk/ws/files/53801282/Aved_re_campaign.pdf}


Wagner, R., Bejdic, J., Windcube + FCR test in Hrgud, Bosnia \& Herzegovina, DTU Wind Energy Report E-0039, March 2014. Online: \small{\url{http://orbit.dtu.dk/en/publications/windcube--fcr-test-at-hrgud\\-bosnia-and-herzegovina(de4febab-3ce2-435c-9aba-\\0583d677981c).html}}


Warmbier,G., Albers, F., Hanswillemenke, K., Verification of wind energy related measurements with a SODAR system, Proc. AWEA Windpower Conference, Pitsburg, 2006.
Online: \small{\url{https://www.environmental-expert.com/articles/verification-of-wind-energy-re\\lated-measurements-with-a-sodar-system-393158}}


Wikipedia, Sigma Coordinate System, Online:                          \small{\url{https://en.wikipedia.org/wiki/Sigma_coordinate_system}}


Wilczak, J. M., and Coauthors, 1995: Contamination of wind profiler data by migrating birds: Characteristics of corrupted data and potential solutions. J. Atmos.  Oceanic Technol., 12, 449­467, doi:10.1175/1520-0426(1995)012<0449:COWPDB>2.0.CO;2., 1995.


Wilczak, J. M., E. E. Gossard, W. D. Neff, and W. L. Eberhard, Ground-based remote sensing of the atmospheric boundary layer: 25 years of progress, Boundary-Layer Meteorol., 78, 321-349, 1996.


Wilczak,J.,Cathy Finley, Jeff Freedman, Joel Cline, Laura Bianco, Joseph Olson, Irina Djalalova, Lindsay Sheridan, Mark Ahlstrom, John Manobianco, John Zack, Jacob R. Carley, Stan Benjamin, Richard Coulter, Larry K. Berg, Jeffrey Mirocha, Kirk Clawson, Edward Natenberg, and Melinda Marquis,  The Wind Forecast Improvement Project (WFIP): A Public–Private Partnership Addressing Wind Energy Forecast Needs. Bull. Amer. Meteor. Soc., 96, 1699–1718, doi: 10.1175/BAMS-D-14-00107.1.,2015. Online: \small{\url{http://journals.ametsoc.org/doi/abs/10.1175/BAMS-D-14-00107.1}}
Wikipedia "Remote Sensing", Online on 5th September 2016: \small{\url{https://en.wikipedia.org/wiki/Remote_sensing}}


W{\"u}rth, I., and Valldecabres, L. and Simon, E. and M{\"o}hrlen, C. and Uzunoglu, B. and Gilbert, C. and Giebel, G. and Schlipf, D. and Kaifel, A., Minute-Scale Forecasting of Wind Power - Results from the collaborative workshop of IEA Wind Task 32 and 36, Energies 2019, 12, 712. Online \url{https://www.mdpi.com/1996-1073/12/4/712}.


Yang, Q., Berg, L.K.,, Pekour, M., Fast, J.D., Newsom, R.K., Evaluation of WRF-Predicted Near-Hub-Height Winds and Ramp Events over a Pacific Northwest Site with Complex Terrain, J. of Appl. Meteo. and Climatology, Vol. 52, p1753-1763, DOI: 10.1175/JAMC-D-12-0267.1, 2013.


Zhong, S., J. D. Fast, and X. Bian, A Case Study of the Great Plains Low-Level Jet Using Wind Profiler Network Data and a High - Resolution Mesoscale Model.  Mon. Wea. Rev.  , 124, pp785–806, 1996.
Online: \small{\url{http://journals.ametsoc.org/doi/abs/10.1175/1520-493(1996)124\%3C0785\%3AACSOTG\%3E2.0.CO\%3B2}}


Zahle, F., & Sørensen, N. N. (2009). Characterisation of the unsteady flow in the nacelle region of a modern wind turbine. In Proc. of the European Wind Energy Conference, 2009.
Online:  
\small{\url{http://orbit.dtu.dk/en/publications/characterization-of-the-unsteady-flow-in-\\the-nacelle-r}egion-of-a-modern-wind-turbine(392515e0-e420-4e3e-a0a7-4979700c5a19)/\\export.html}

\setlength{\parindent}{0in}
%\end{thebibliography}